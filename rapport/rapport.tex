\documentclass[12pt, twoside, czech]{report}
\usepackage[T1]{fontenc}
\usepackage[utf8]{inputenc}
	\makeatletter
		\def\UTFviii@defined#1{%
		\ifx#1\relax
			!!FIXME%
		\else
			\expandafter#1%
		\fi
	}
	\makeatother

\usepackage[french]{babel}
\usepackage{amsfonts,amsmath,amssymb}
\usepackage{lastpage}
\usepackage{fancyhdr}
%\usepackage{scrlayer-scrpage}
%\usepackage{blindtext}
\usepackage{graphicx}
\usepackage[left=3cm,right=2.5cm,top=2.5cm,bottom=2.5cm]{geometry}
\geometry{verbose}
\usepackage{color}
\usepackage[dvipsnames]{xcolor}
\usepackage{colortbl}
\usepackage{pifont}
\usepackage{times}
\usepackage{eso-pic}
\usepackage{setspace}
\usepackage{textcomp} % pour le drege


\usepackage{tocloft}
\usepackage{tikz}
\usepackage{xurl}
\usepackage[bookmarksopen, bookmarksdepth=2, colorlinks=false, urlcolor=teal, breaklinks=true]{hyperref}

\usepackage[most]{tcolorbox}
\usepackage{array}
\usepackage{color}
\usepackage{fancyhdr}
\usepackage{booktabs} % Pour l'affichage des tableaux toprule
\usepackage{tabularx}

\fancypagestyle{fancy}{
    \fancyhf{} 
    \fancyfoot[C]{\thepage}
    \setlength{\footskip}{45pt} 
    \renewcommand{\headrulewidth}{0pt}
    \renewcommand{\footrulewidth}{0pt}


    }
 

\pagestyle{fancy} 	

\makeatletter
\def\thickhrulefill{\leavevmode \leaders \hrule height 1ex \hfill \kern \z@}
\def\@makechapterhead#1{%
  %\vspace*{50\p@}%
  \vspace*{10\p@}%
  {\parindent \z@ \centering \reset@font
        \thickhrulefill\quad
       
        \scshape \@chapapp{} \thechapter
        \quad \thickhrulefill
        \par\nobreak
        \vspace*{10\p@}%
        \interlinepenalty\@M
        \hrule
        \vspace*{10\p@}%
        \Huge \bfseries #1\par\nobreak
        \par
        \vspace*{10\p@}%
        \hrule
    %\vskip 40\p@
    \vskip 50\p@
  }}
\def\@makeschapterhead#1{%
  %\vspace*{50\p@}%
  \vspace*{10\p@}%
  {\parindent \z@ \centering \reset@font
        \thickhrulefill
        \par\nobreak
        \vspace*{10\p@}%
        \interlinepenalty\@M
        \hrule
        \vspace*{10\p@}%
        \Huge \bfseries #1\par\nobreak
        \par
        \vspace*{10\p@}%
        \hrule
    %\vskip 40\p@
    \vskip 50\p@
  }}


\setcounter{secnumdepth}{3}

\urlstyle{tt}

\hypersetup{pdfborder={0 0 0}}
\renewcommand{\cftsecleader}{\hfill}

%\renewcommand{\baselinestretch}{1.5}

\AtEndDocument{\label{lastpage}}
\begin{document}


\begin{titlepage}
		
		\begin{center}\Large
			\textsc{République De Côte D'ivoire}
		\end{center}
		
		\begin{figure}[!h]
			\begin{center}
				\includegraphics[scale=.35]{armoiries.png}
			\end{center}
		\end{figure}
		
		\vspace*{0.05cm}
		
		\begin{center}\Large
			 \textsc{International Data Science Institute}
		\end{center}		
		
		\begin{figure}[!h]
			\begin{center}
				\includegraphics[scale=.08]{logoINP.png} \hspace*{3cm} \includegraphics[scale=.45]{images.png} \hspace*{3cm} \includegraphics[scale=.15]{POLYTECHNIQUE-IP_PARIS.png}
			\end{center}
		\end{figure}
		
		\begin{center}\Large
			 PROJET DE TEXT MINING
		\end{center}
		
		\vspace*{0.5cm}
		
		\parindent=0cm
		\rule{\linewidth}{1mm}
		\baselineskip=0.1cm
		\rule{\linewidth}{0.4mm}
		
		\begin{center}
   			\Large{Analyse Comparative des Discours Sanitaires} \\
		\end{center}
		
		\rule{\linewidth}{0.4mm}
		\baselineskip=-0.1cm
		\rule{\linewidth}{1mm}
		
		
		\begin{center}
			\textsc{Réalisé par :} \\ 
			\vspace*{0.3cm}
			\bfseries{ADJINDA Adékin Olakêmi Lucia \\
			\vspace*{0.2cm}
			COULIBALY Segnindenin Oumar \\
			\vspace*{0.2cm}
			NIADA Ulrich Boris \\
			\vspace*{0.2cm}
			SAWADOGO Amed}
		\end{center}
		
	 \vspace*{0.3cm}
		
		\begin{center}
   			 \textsc{Période :}\\
   			 \vspace*{0.3cm}
   			 Avril - Juin 2025
		\end{center}
		
		\vspace*{0.3cm}
		
		\begin{center}
   			 \textsc{Enseignant :}\\
   			 \vspace*{0.3cm}
   			 \bfseries{\textsc{Dr TANOH Lambert}} \\
		\end{center}
		
		\begin{center}
   			 Directeur de l'IDSI \\
   			 \vspace*{0.2cm}
   			 DSI de l'INPHB
		\end{center}
		
		\vspace*{0.3cm}
		
		\begin{center}
			\textsc{Master 1 Data Science, Big Data \& Intélligence Artificielle}
		\end{center}
		
		
		\vspace*{0.7cm}
		
		\begin{flushright}
			\textbf{\textit{Année académique 2024-2025}}
		\end{flushright}
		
		
\end{titlepage}


\tableofcontents
\newpage

% ========================
\section{Résumé exécutif}
Ce rapport analyse dans quelle mesure les priorités sanitaires communiquées par l’Organisation Mondiale de la Santé (OMS) – Région Afrique sont reprises, reformulées ou ignorées par Forbes Afrique, média économique majeur du continent.

L’analyse repose sur un pipeline complet de \textit{text mining} incluant :
\begin{itemize}
    \item extraction de thématiques (LDA) ;
    \item similarité lexicale et sémantique (TF-IDF, SBERT) ;
    \item analyse du sentiment ;
    \item analyse du \textit{framing} éditorial ;
    \item extraction d'entités nommées (NER) ;
    \item visualisation UMAP ;
    \item création d’un dashboard interactif.
\end{itemize}

Les résultats montrent un écart important entre les enjeux de santé publique (OMS) et la vision économique centrée sur l’innovation (Forbes Afrique).

% ========================
\section{Introduction}
L'Afrique fait face à de nombreux défis sanitaires : épidémies récurrentes, mortalité infantile élevée, faible couverture vaccinale, infrastructures insuffisantes.

L’OMS documente ces problématiques dans ses communications officielles. Forbes Afrique, en revanche, focalise sa couverture sur l’économie, l’innovation et l’entrepreneuriat.

\textbf{Objectif général :} mesurer le degré de convergence entre ces deux visions à travers une analyse textuelle approfondie.

\textbf{Questions de recherche :}
\begin{itemize}
    \item Quels sont les thèmes prioritaires de l’OMS ?
    \item Dans quelle mesure Forbes couvre-t-il ces thèmes ?
    \item Quelle est la proximité sémantique OMS–Forbes ?
    \item Comment les deux sources diffèrent-elles en termes de sentiment et d’angle éditorial (\textit{framing}) ?
\end{itemize}

% ========================
\section{Corpus et Description des Données}

\subsection{Corpus OMS Afrique}
\begin{itemize}
    \item $\sim 60$ articles
    \item Langue : principalement français
    \item Types : communiqués, bulletins épidémiologiques, rapports terrain
    \item Thèmes récurrents : épidémies, vaccination, mortalité maternelle, urgences sanitaires
\end{itemize}

\subsection{Corpus Forbes Afrique}
\begin{itemize}
    \item $\sim 50$ articles
    \item Langue : français
    \item Types : interviews, reportages économiques, analyses de marché
    \item Thèmes : innovation, biotech, start-ups santé, financements
\end{itemize}

% ========================
\section{Méthodologie}

\subsection{Nettoyage et Prétraitement}
\begin{itemize}
    \item normalisation : minuscules, suppression URLs, accents ;
    \item suppression des mots outils FR/EN ;
    \item tokenisation ;
    \item lemmatisation/stemming.
\end{itemize}

\subsection{Modélisation du Texte}
\begin{itemize}
    \item \textbf{TF-IDF} : comparaison lexicale ;
    \item \textbf{SBERT} (all-mpnet-base-v2) : représentation sémantique ;
    \item \textbf{UMAP} : réduction dimensionnelle.
\end{itemize}

\subsection{Extraction des Thèmes OMS (LDA)}
\begin{itemize}
    \item 8 thèmes extraits automatiquement ;
    \item servent de référence pour mesurer la couverture Forbes.
\end{itemize}

\subsection{Analyse du Sentiment et Framing}
\begin{itemize}
    \item modèle Transformer multilingue ;
    \item labels : POS / NEG / NEU ;
    \item \textit{framing} : économique, sanitaire, mixte, neutre.
\end{itemize}

\subsection{Extraction d’Entités (NER – spaCy)}
Labels utilisées : PER, ORG, LOC, GPE, MISC.

% ========================
\section{Résultats}

\subsection{Thèmes OMS Identifiés}
\begin{itemize}
    \item Épidémies (Ebola, choléra)
    \item Vaccination
    \item Santé maternelle
    \item Accès aux soins
    \item Surveillance épidémiologique
    \item Tuberculose
    \item Urgences humanitaires
    \item Renforcement des systèmes de santé
\end{itemize}

\textbf{Tableau 1 :} Résumé des 8 thèmes OMS extraits.  
(À compléter avec votre tableau)

% ========================
\subsection{Couverture Forbes des Thèmes OMS}

\textbf{À insérer : Barplot couverture Forbes → thèmes OMS}



\textbf{Interprétation (pré-rédigée) :}  
Forbes couvre principalement les thèmes liés à l’innovation, aux investissements et au marché de la santé.  
Les thématiques critiques (choléra, crises humanitaires, mortalité maternelle) sont très peu relayées.

% ========================
\subsection{Analyse UMAP – Proximité Sémantique}



\begin{figure}[!htbp]
    	\centering
    	\includegraphics[width=0.85\textwidth]{figures/umap_sources.png
   	\caption{Projection UMAP des articles OMS vs Forbes}
 
\end{figure}


\textbf{Interprétation (pré-rédigée) :}  
L’UMAP montre deux clusters distincts :
\begin{itemize}
    \item OMS : discours sanitaire, urgence, santé publique ;
    \item Forbes : innovation, économie, financement.
\end{itemize}
Quelques points hybrides apparaissent sur les sujets de vaccination et de systèmes de santé.

% ========================
\subsection{Analyse du Sentiment}



\textbf{Interprétation :}  
OMS adopte un ton neutre ou négatif (crises sanitaires), tandis que Forbes est majoritairement positif (innovations, success stories).

% ========================
\subsection{Analyse des Entités et Wordclouds}



\textbf{Interprétation :}  
OMS : crise, maladie, épidémie, vaccination.  
Forbes : innovation, technologie, entreprise, marché.



% ========================
\section{Conclusion}
Cette étude met en évidence un écart significatif entre les priorités sanitaires de l’OMS et les priorités éditoriales de Forbes Afrique.

\textbf{Résumé des divergences majeures :}
\begin{itemize}
    \item Santé publique vs Santé économique ;
    \item urgence sanitaire vs innovation ;
    \item institutions publiques vs acteurs privés.
\end{itemize}

\textbf{Perspectives (à compléter) :}
\begin{itemize}
    \item Analyse temporelle ;
    \item Analyse par pays ;
    \item Inclusion d’autres médias africains.
\end{itemize}

\end{document}
